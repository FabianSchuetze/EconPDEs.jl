\documentclass[english]{article}
\usepackage[T1]{fontenc}
\usepackage[latin9]{inputenc}
\usepackage{color}
\usepackage{amsmath}
\usepackage{amssymb}
\usepackage{esint}
\usepackage{babel}
\usepackage[round]{natbib}
\begin{document}

\title{Algorithm to Solve PDE Models}
\author{\large{\textsc{Matthieu Gomez \thanks{I thank Valentin Haddad, Ben Moll, and Dejanir Silva for useful discussions.}}}}
\date{\today}
\maketitle
This note details an algorithm with excellent convergence properties to solve PDEs associated with economic models in continuous time.

\section{Solving Finite Difference Schemes}

\subsection{How $\Psi tc$ works}
Denote $Y$  the solution and denote $F(Y) = 0$ the finite difference scheme corresponding to a model. On can solve for $Y$ using one of the two methods:

\begin{enumerate}
	\item Non linear solver for $F (Y ) = 0$. Updates take the form
	\begin{align*}
		0 &= F(y_{t}) + J_{F}(y_t) (y_{t+1} - y_t)
	\end{align*}
	The method converges only if the initial guess is sufficiently close to the solution. \footnote{For instance, see \citet{campbell1999force}.}
	\item ODE solver for $F(Y) = \dot{Y}$. The solution of $F(Y)=0$ is obtained with $T\rightarrow +\infty$. \footnote{For instance, see \citet{ditellabalance}.}
	With a simple explicit Euler method, updates take the form
	\begin{align*}
		0&= F(y_t) - \frac{1}{\Delta} (y_{t+1} -y_{t})
	\end{align*}
	Convergence conditions are given by the Barles-Souganadis theorem. Explicit schemes usually don't satisfy them.
\end{enumerate}
I propose to use a fully implicit Euler method, which has better convergence properties than explicit schemes.  Updates take the form 
\begin{align*}
	\forall t \leq T \hspace{1cm} 0&= F(y_{t+1}) - \frac{1}{\Delta}(y_{t+1} -y_{t})
\end{align*}
Each time step is a non linear equation, which I solve using a Newton-Raphson method. These inner iterations take the form
\begin{align*}
	\forall i \leq I \hspace{1cm}	0 &= F(y_{t}^i) - \frac{1}{\Delta}(y_{t}^{i} -y_{t}) + (J_{F}(y_t^i) -  \frac{1}{\Delta})(y^{i+1}_{t} - y_t^i)
\end{align*}
As pointed above, the Newton-Raphson method converges when $y_t$ is sufficiently close to $y_{t+1}$. Therefore I decrease $\Delta$ until the inner Newton-Raphson method converges.\footnote{However, I cannot prove the convergence of the overall scheme when $\Delta$ depends on the step. The Barles-Souganadis theorem ensures that the implicit Euler scheme converges only with $\Delta$ fixed. 	\citet{kelley1998convergence} provides the formal proof of convergence for the algorithm with some supplementary assumptions on $F$}.\\
I accomodate algebraic equations by setting $\Delta = +\infty$ for these equations. This ensures that the PDEs are solved backward on a path that always satisfies the algebraic constraints.

How does the method relate to the two algorithms seen above? When $I =1$ (i.e. with only one inner iteration) the update can be seen as the sum of a Newton-Raphson and an explicit time step
\begin{align*}
	\forall t \leq T \hspace{1cm} 0&= F(y_{t+1}) + (J_{F}(y_t) - \frac{1}{\Delta})(y_{t+1} -y_{t})
\end{align*}
Alternatively, the method can be also be seen as a dampened Newton-Raphson algorithm. As in the Levenberg-Marquardt method, the diagonal of the Jacobian is modified until the algorithm gets close to the solution.

The algorithm usally converges quickly: the convergence is quadratic around the solution, since the algorithm follows the Newton-Rapshon method around the solution.

\subsection{Related Methods}
	
\begin{itemize}
	\item 
	With $I=1$ and constant $\Delta$, we obtain the method in \citet{achdou2014heterogeneous} for a partial equilibrium consumption / saving  problem with separable preference. I've found that allowing $\Delta$ to change  and a $I > 1$ are important for the robustness of the algorithm. Moreover, $\Psi tc$ handles systems with implicit jacobians and algebraic equations.

	\item 
	A similar algorithm is used in Fluid Dynamics. In this context, it is called Pseudo-Transient Continuation (denoted $\Psi tc$). 
\end{itemize}

\section{Writing Finite Difference Schemes}
\begin{itemize}
	\item 
	Write the finite difference scheme so that the implicit Euler method satisfies the convergence conditions of Barles-Souganadis theorem (as much as possible). In particular,
	\begin{itemize}
		\item Upwind first derivatives to make the scheme monotonous (for instance see Achdou, Han, Lasry, Lions (2016))
		\item Write the function $F$ so that $\dot{Y}$ would appear as such (i.e. not multiplied by some parameters). For instance, a typical PDE for the price dividend ratio should be written
		\begin{align*}
			0 &= p (\frac{1}{p} + E[\frac{dD}{D}] + E[\frac{dp}{p}] + \sigma[\frac{dp}{dp}]\sigma[\frac{dD}{dD}] - r - \kappa(\sigma[\frac{dp}{dp}] + \sigma[\frac{dD}{D}]))
		\end{align*}
	\end{itemize}
	\item When solving for multiple functions, use the same economic quantities across the different equations. This ensures that the time step is comparable across different equations. For instance use the wealth / consumption of each agent in heterogeneous agent models.
\end{itemize}
\bibliography{bib}
\bibliographystyle{aer}

\end{document}
