\documentclass[english]{article}
\usepackage[T1]{fontenc}
\usepackage[latin9]{inputenc}
\usepackage{color}
\usepackage{amsmath}
\usepackage{amssymb}
\usepackage{esint}
\usepackage{babel}
\usepackage[round]{natbib}
\begin{document}
	\title{Solving Economic Models in Continuous Time using the Pseudo-Transient Method}
	\author{\large{\textsc{Matthieu Gomez \thanks{I thank Valentin Haddad, Ben Moll, and Dejanir Silva for useful discussions.}}}}
	\date{\today}
	\maketitle
	This note details how to solve systems of PDEs + algebraic equations associated with economic models in continuous time.

	\paragraph{Algorithm.}
	I propose to solve the non linear system corresponding to economic models using the Pseudo-Transient Continuation (denoted $\Psi tc$) method. This method is used in the fluid dynamics literature. 
	\par Formal conditions for the convergence of the algorithm are given in  \citet{kelley1998convergence}, but I now derive an intuition for the algorithm.\par
	Denote $Y$  the solution and denote $F(Y) = 0$ the finite difference scheme corresponding to a model. The existing literature in economics solves for $Y$ using using one of the two methods:

	\begin{enumerate}
		\item Non linear solver for $F (Y ) = 0$. Updates take the form
		\begin{align*}
			0 &= F(y_{t}) + J_{F}(y_t) (y_{t+1} - y_t)
		\end{align*}
		The method converges only if the initial guess is sufficiently close to the solution.\footnote{See, for instance, \citet{campbell1999force}.}
		\item ODE solver for $F(Y) = \dot{Y}$. The solution of $F(Y)=0$ is obtained with $T\rightarrow +\infty$. \footnote{See, for instance, \citet{ditellabalance}, \citet{silva2015risk}.}
		With a simple explicit Euler method, updates take the form
		\begin{align*}
			0&= F(y_t) - \frac{1}{\Delta} (y_{t+1} -y_{t})
		\end{align*}
		Convergence conditions are given by the Barles-Souganadis theorem. Explicit schemes usually don't satisfy them.
	\end{enumerate}
	Rather than one of these two methods, I propose to solve for $Y$ using fully implicit Euler method.  Updates take the form 
	\begin{align*}
		\forall t \leq T \hspace{1cm} 0&= F(y_{t+1}) - \frac{1}{\Delta}(y_{t+1} -y_{t})
	\end{align*}
	Each time step is a non linear equation, which I solve using a Newton-Raphson method. These inner iterations take the form
	\begin{align*}
		\forall i \leq I \hspace{1cm}	0 &= F(y_{t}^i) - \frac{1}{\Delta}(y_{t}^{i} -y_{t}) + (J_{F}(y_t^i) -  \frac{1}{\Delta})(y^{i+1}_{t} - y_t^i)
	\end{align*}
	These inner iterations converge as long as $y_t$ is sufficiently close to $y_{t+1}$. Therefore I decrease $\Delta$ until the inner Newton-Raphson method converges.\par
	How does the method relate to the two algorithms seen above? When $I =1$ (i.e. with only one inner iteration) the update can be seen as the sum of a Newton-Raphson and an explicit time step
	\begin{align*}
		\forall t \leq T \hspace{1cm} 0&= F(y_{t+1}) + (J_{F}(y_t) - \frac{1}{\Delta})(y_{t+1} -y_{t})
	\end{align*}
	The step becomes close to a pure Newton-Rapshon step as $\Delta \rightarrow + \infty$. Therefore, the convergence is quadratic around the solution. \par
	I accommodate algebraic equations by setting $\Delta$ to $+\infty$ for the coordinates  of $F$ that correspond to algebraic equations. This ensures that the PDEs are solved backward on a path that always satisfies the algebraic constraints.\par
	\paragraph{Relation with other methods.} The algorithm with $I=1$ and $\Delta$ constant corresponds to \citet{achdou2014heterogeneous}, an algorithm for PE models. Allowing $I > 1$ and adjusting $\Delta$  are important to ensure convergence in GE models.

	\paragraph{Writing Finite Difference Schemes.}
	I now give some heuristics to write correctly the finite difference scheme $F$ given the system of PDEs characterizing some economic model.\par
	The goal is to write the function $F$ so that the implicit Euler method satisfies the convergence conditions of Barles-Souganadis theorem as much as possible. In particular,
	\begin{itemize}
		\item Upwind first derivatives to make the scheme monotonous (for instance see \citet{achdou2014heterogeneous})
		\item Write the function $F$ so that $\dot{Y}$ would appear as such (i.e. not multiplied by some parameters). For instance, a typical PDE for the price dividend ratio should be written
		\begin{align*}
			0 &= p (\frac{1}{p} + E[\frac{dD}{D}] + E[\frac{dp}{p}] + \sigma[\frac{dp}{dp}]\sigma[\frac{dD}{dD}] - r - \kappa(\sigma[\frac{dp}{dp}] + \sigma[\frac{dD}{D}]))
		\end{align*}
		\item  When solving a system of PDEs, use the same economic quantities across the different equations. This ensures that the time step is comparable across different equations. For instance use the wealth / consumption of each agent in heterogeneous agent models.
	\end{itemize}
	\bibliography{bib}
	\bibliographystyle{aer}
\end{document}
