\documentclass[english]{article}
\usepackage[T1]{fontenc}
\usepackage[latin9]{inputenc}
\usepackage{color}
\usepackage{amsmath}
\usepackage{amssymb}
\usepackage{esint}
\usepackage{babel}
\usepackage[capitalise, noabbrev]{cleveref}
\usepackage[round]{natbib}
\begin{document}
	\title{Solving Economic Models in Continuous Time using the Pseudo-Transient Method}
	\author{\large{\textsc{Matthieu Gomez \thanks{I thank Valentin Haddad, Ben Moll, and Dejanir Silva for useful discussions.}}}}
	\date{\today}
	\maketitle
	This note details how to solve systems of PDEs + algebraic equations associated with economic models in continuous time.

	\paragraph{Intuition.}
Denote $Y$  the solution and denote $F(Y)$ the finite difference scheme corresponding to a model. The goal is to find $Y$ such that $F(Y) = 0$ The existing literature in economics solves for $Y$ using using one of the two methods:

	\begin{enumerate}
		\item Non linear solver for $F (Y ) = 0$. A Newton-Raphson update takes the form
		\begin{align}\label{1}
			0 &= F(y_{t}) + J_{F}(y_t) (y_{t+1} - y_t)
		\end{align}
		The method converges only if the initial guess is sufficiently close to the solution.\footnote{See, for instance, \citet{campbell1999force}.}
		\item ODE solver for $F(Y) = \dot{Y}$. The solution of $F(Y)=0$ is obtained with $T\rightarrow +\infty$. \footnote{See, for instance, \citet{ditellabalance}, \citet{silva2015risk}.}
		With a simple explicit Euler method, updates take the form
		\begin{align}\label{2}
			0&= F(y_t) - \frac{1}{\Delta} (y_{t+1} -y_{t})
		\end{align}
		Convergence conditions are given by the Barles-Souganadis theorem. Explicit schemes usually don't satisfy them.
	\end{enumerate}
	Rather than one of these two methods, I propose to solve for $Y$ using fully implicit Euler method.  Updates now take the form 
	\begin{align*}
		\forall t \leq T \hspace{1cm} 0&= F(y_{t+1}) - \frac{1}{\Delta}(y_{t+1} -y_{t})
	\end{align*}
	Each time step now requires to solve a non linear equation. I solve this non linear equation using a Newton-Raphson method. These inner iterations take the form
	\begin{align}\label{inner}
		\forall i \leq I \hspace{1cm}	0 &= F(y_{t}^i) - \frac{1}{\Delta}(y_{t}^{i} -y_{t}) + (J_{F}(y_t^i) -  \frac{1}{\Delta})(y^{i+1}_{t} - y_t^i)
	\end{align}
	We know that the Newton-Raphson method converges if the initial guess is close enough to the solution. Since $y_{t}$ converges towards $y_{t+1}$ as $\Delta$ tends to zero, one can always choose $\Delta$ low enough so that the inner steps converge.\par
	I adjust $\Delta$ as follows. If the inner iterations do not converge, I decrease $\Delta$. When the inner iteration converges, I increase $\Delta$. \par
	The update \cref{inner} can be see as weighted average of the Newton-Raphson step \cref{1} and of the explicit Euler step \cref{2}.  After a few sucessful implicit time steps, $\Delta$ is large and therefore the algorithm becomes like Newton-Rapshon. In particular, the convergence is quadratic around the solution. \par
	I accommodate algebraic equations by setting $\Delta$ to $+\infty$ along these coordinates.
	\paragraph{Relation with other methods.} 	The method actually corresponds to a method used in he fluid dynamics literature. In this context, it is called the Pseudo-Transient Continuation method, and is denoted $\Psi tc$. Formal conditions for the convergence of the algorithm are given in  \citet{kelley1998convergence}.\par
	The algorithm with $I=1$ and $\Delta$ constant corresponds to \citet{achdou2014heterogeneous}. They prove the convergence of this algorithm for models in partial equilibrium. Allowing $I > 1$ and adjusting $\Delta$  are important to ensure convergence in general equilibrium, which are non linear.

	\paragraph{Writing Finite Difference Schemes.} It is important to write correctly the finite difference scheme $F$. A good heuristic is to write it so that the implicit Euler method satisfies the convergence conditions of Barles-Souganadis theorem. In particular,
	\begin{itemize}
		\item Upwind first derivatives to make the scheme monotonous (for instance see \citet{achdou2014heterogeneous})
		\item Write each PDE as a no arbitrage condition for a particular asset. Denoting $p$ the price dividend ratio:
		\begin{align*}
			0 &= p (\frac{1}{p} + E[\frac{dD}{D}] + E[\frac{dp}{p}] + \sigma[\frac{dp}{dp}]\sigma[\frac{dD}{dD}] - r - \kappa(\sigma[\frac{dp}{dp}] + \sigma[\frac{dD}{D}]))
		\end{align*}
	\end{itemize}
	\bibliography{bib}
	\bibliographystyle{aer}
\end{document}
